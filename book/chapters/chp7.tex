\chapter{Concluding remarks and outlook}

Physics-based modelling of brain mechanics,  informed by multi-modal
imaging data, is an exciting and still very much frontier research
area. Through this introductory and hands-on text, we have aimed at
presenting sufficient yet an accessible amount of material to place
the reader near the research front and to establish a solid foundation
for further scientific investigations.

The MRI data to finite element pipeline presented here can be used for
further simulations of solute transport within the brain, and
importantly, provides a basis for more complex simulation
scenarios. Our techniques for creating finite element meshes from T1
MR images can be equally applicable for numerical simulation of the
brain as an elastic medium in the context of traumatic brain injuries,
as a poroelastic medium for studying neurological disorders or stroke,
or as an electrical medium for studying the propagation of epileptic
seizures. Similarly, anisotropy data extracted from DTI can inform
diffusion tensors (as here), or the permeability tensor in the context
of brain fluid movement, the conductivity tensor in the context of
brain electrophysiology, or possibly the compliance tensor(s) in the
context of brain elasticity. While such physical models have not been
considered here, the meshes and finite element simulation platform
FEniCS extend readily to these contexts.

The brain does not exist in isolation, but is tightly coupled to its
local environment, including the surrounding CSF, vasculature,
membranes and spinal cord. The components presented here can be used
as building blocks for computational modelling of the brain and its
local environment, but geometrical, numerical, and computational
challenges remain for this non-trivial multi-physics setting.

Over the next decades, we envision that mathematics and numerical
computations could play a crucial role in gaining new insight into the
mechanisms and physiological processes of the human brain. Indeed,
clinicians and experimentalists express a need for modelling and
simulation as an alternative avenue of investigation to alleviate
fundamental limitations in traditional techniques. New understanding
of physiology and pathology could then ultimately pave the way for new
diagnostics and treatments of a range of brain disorders and diseases
-- with immense scientific and societal impact.

