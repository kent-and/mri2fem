\chapter{Introduction}
\label{chp:chp1}

Our brain is our most precious, yet most mysterious organ. It consists
of near 100 billion neurons, each neuron typically having 10,000 connections
of lengths of up to a meter. As such, it is a intricate web in which
and with which we experience the world. In addition to neurons,
the brain consists of about the same number of glial cells, around 700
kilometers of blood vessels, and the extracellular matrix, and is
surrounded by clear water-like cerebrospinal fluid, which
together all work to maintain the delicate neurons' environment in a
healthy state. At the whole-organ level, this is already an enormous amount of 
complexity yet we still have not added several factors; such as the electrical 
impulses between neurons or the complex chemical signaling pathways that help 
to maintain homeostasis.  Due to the innate micro-scale complexity of the brain,  
a natural approach, attempting to understand the brain's physiology
and function, is homogenized, continuum-based modeling;  here, the 
focus is on modeling the large-scale behavior arising from the aggregate of 
small-scale contributions. 

While continuum-based brain modeling has many applications, our
motivation for writing this book and the corresponding software tools
comes from recent theories concerning the restorative mechanisms of
sleep. Recent theories consider the brain a porous and elastic
(poroelastic) medium, where the elastic medium consists of the cells
and the fluid-filled pores are the extra-cellular matrix (of course, 
hyper- and viscoelastic materials~\cite{goriely2015mechanics,
  budday2019fifty} could also be considered). In this setting, a
paradigm shift was introduced by the glymphatic theory, proposed and
developed over the last eight years by the Nedergaard
group~\cite{iliff2012paravascular}. The glymphatic theory proposes
that extra-cellular diffusion, as described in the seminal work of
Sykov{\'a} and Nicholson~\cite{sykova2008diffusion}, is not sufficient
to explain the fundamental transport processes in the brain. In
particular, pressure-driven convective flow is hypothesized to wash
away the larger molecules of the metabolic waste products produced
during the day~\cite{iliff2012paravascular, jessen2015glymphatic,
  xie2013sleep}. Such metabolic waste proteins are observed to
accumulate in patients with neurodegenerative diseases such as
Alzheimer's or Parkinson's disease.

This topic has received a great deal of attention from the modeling
community concerning the mechanisms at the microscopic level,
e.g.~\cite{aldea2019cerebrovascular, daversin2020mechanisms,
  diem2017arterial, holter2017interstitial, ray2019analysis,
  sharp2019dispersion, smith2017test}, to name but a few. However,
very few works address the macroscopic level and how to employ
modeling in a patient-specific manner, see e.g.,~\cite{chou2016fully,
  lee2019mixed}. Mesh generation based on medical images and typically
integration of images of different types is crucial to achieve
patient-specific assessment. Our intention with this book is to equip
the reader with software tools to perform such types of studies.

Many other applications involve continuum-based models of
the brain's physiology.  For instance, alternative macroscopic
theories involving the prion-related development of Alzheimer's
disease have been proposed~\cite{fornari2019prion,
  kevrekidis2020anisotropic}. Another interesting observation is the
fact that astronauts often experience visual impairments and are at
risk of developing early dementia as a consequence of their periods in
low or zero gravity. The reason seems to be intracranial pressure
changes and a shift in fluid volumes in intracranial
compartments~\cite{alperin2018spaceflight}.  Another well-known
computational modeling problem is motivated by epilepsy. A hot
research topic in this area is the inverse problem of
electroencephalography (EEG) of determining the source of an epileptic
seizure via an elliptic PDE~\cite{grech2008review}.

\begin{figure}
  \begin{center}
  \includegraphics[height=2.3cm]{./graphics/chp1/T1-image-rot-white}
  \includegraphics[height=2.3cm]{./graphics/chp1/ernie-volume-64}
  \includegraphics[height=2.3cm]{./graphics/chp1/soltn-t30-crop}
  \hfill
  \end{center}
  \caption{Going from magnetic resonance (MR) images of a human brain
    to a numerical simulation of a biophysical phenomenon. From left
    to right: (a) An MR image of a human brain viewed along the axial
    direction, (b) a finite element mesh extracted from the MR image,
    (c) a snapshot of a tracer distribution simulation over this
    mesh. MR image types are discussed in Chapter~\ref{chp:chp2}}
  \label{fig:chp1:pipeline}
\end{figure}

This book focuses on creating the basic foundation, of computational resources, 
necessary to enable continuum-based modeling of the human brain.  Though we 
don't focus explicitly on the exciting multiphysics applications mentioned 
above, the approach discussed here is generalizable to multiphysics problems.  
You, the reader of this book, will learn how to formulate, set-up and implement 
mathematical and computational models of brain biophysics in patient-specific 
geometries using finite element simulations and MR images (see Figure
\ref{fig:chp1:pipeline}). We will use the evolution and distribution
of a solute concentration due to diffusion as a model problem, and
increase the complexity of the data and techniques involved in the
course of the book. Of course, the process involves several challenges
and pitfalls which will be outlined.

\section{A model problem}

\index{diffusion equation}
Suppose that we aim to study the diffusion of a solute concentration
in a region of the brain. The region $\Omega$ could represent the left
brain hemisphere or a smaller region such as the hippocampus, while
the concentration $u$ could represent an injected tracer used in
imaging (such as gadobutrol~\cite{ringstad2018brain} or
dextran~\cite{iliff2013cerebral}) or possibly a metabolic waste
protein associated with neurodegenerative disease such as
amyloid-$\beta$ or tau. We can describe this model problem by a
time-dependent partial differential equation (PDE): find the
concentration $u = u(t, x)$ at spatial points $x \in \Omega$ and time
points $t > 0$ such that
\begin{subequations}
  \label{eq:diffusion}
  \begin{align}
    \label{eq:diffusion:a}
    u_t - \Div D \Grad u &= f &&\text{ in } (0, T] \times \Omega, \\
    \label{eq:diffusion:b}
    u &= u_d && \text{ on } (0, T] \times \partial \Omega, \\
    \label{eq:diffusion:c}
    u(0, \cdot) &= u_0 && \text{ in } \Omega.
  \end{align}
\end{subequations}
In the diffusion equation~\eqref{eq:diffusion:a}, $u$ is the unknown
field, while $D$ is the diffusion tensor coefficient and $f$
represents any source or sink for the concentration within the
domain. The subscript $t$ denotes the time derivative, $\Div$
represents the divergence and $\Grad$ the spatial gradient. The second
equation~\eqref{eq:diffusion:b} gives a boundary condition: the
function $u_d$ represents a known distribution of the concentration on
the boundary $\partial \Omega$ for all times. The third
equation~\eqref{eq:diffusion:c} gives an initial condition for the
solute concentration: the function $u_0$ represents the known initial
concentration distribution throughout $\Omega$ at $t=0$. The combined
problem \eqref{eq:diffusion} is a complete initial boundary-value
problem and will be our model problem.


\section{On reading this book}
This text does not assume that the reader is well versed in anatomy or in 
neuroscience.  In fact, the bulk of the anatomical familiarity needed to 
successfully follow along with this text is covered in 
Chapter~\ref{sec:chp2:anatomy}.  We have also made liberal use of footnotes and 
citations to inform the reader of additionally interesting, or contextually 
useful, anatomical or physiological details. This text does, however, assume a 
basic knowledge of PDEs. For instance, the diffusion equation 
\eqref{eq:diffusion} is a classical continuum-based PDE with well-known 
behavior in both the mathematical and numerical sense. The reader unfamiliar 
with this equation is advised to first consult an introductory text on solving 
PDEs using the finite element
method~\cite{gockenbach2006understanding,
  langtangen2016solving, langtangen2019introduction,tveito2004introduction}.

The reader is assumed to be comfortable executing commands from a
command line in a terminal window (also canonically referred
to as a command window or command prompt). Terminal commands
will throughout be formatted as:
\terminal{\$ cd ..}
\noindent Commands at the operating-system (OS) level, such as that above,
can differ from OS to OS, and we mainly demonstrate Linux commands
here.

We also assume the reader is familiar with the fundamentals of the
Python programming language or, alternatively, can understand the
syntax well enough to follow the source code that appears throughout;
we will not use any advanced Python programming techniques. We use
Python 3 throughout, and we must therefore make sure that we have Python
version 3.0 or any later version installed. You can check
your Python version by either of the following terminal commands:
\terminal{\$ python {\ddash}version \\
\$ python3 {\ddash}version
}

We will use the Python interface to the FEniCS Project finite
element software~\cite{alnaes2015fenics}, and we assume that the
reader is familiar with the material covered by the FEniCS
tutorial~\cite{langtangen2016solving}.

\section{Data sets and scripts}

\index{mri2fem data sets and scripts}

The data sets and scripts used and described in this book are openly
available and associated with its Zenodo community:
\url{https://zenodo.org/communities/mri2fem/}. 
\begin{itemize}
\item
  The book data set, including MR images, can be downloaded from \\
  \url{http://doi.org/10.5281/zenodo.4386986}~\cite{kent_andre_mardal_2020_4386986}.
\item
  The book scripts can be downloaded from \\
  \url{http://doi.org/10.5281/zenodo.4386998}~\cite{kent_andre_mardal_2020_4386998}.
\item 
  A git repository containing the book and its scripts can be found 
  at: \\ \url{https://github.com/kent-and/mri2fem}.
\end{itemize}
We strongly recommend that you download and unpack these materials
before reading further. We expect to update the Zenodo community with
script updates, updated installation guides and further material as
needed.
 
\section{Other software}
We will use a number of external tools in this book. Most of these
tools are available for a number of operating systems, with separate
installation instructions and dependencies for each system. For the
key external tools, in this book, we provide installation instructions for Linux
Ubuntu (version 20.04, but earlier or later versions might also work
fine). Whenever we refer to an Ubuntu specific terminal command, we
format it as follows:
\ubuntu{\$ sudo apt-get install ... }
\noindent We note that, before installing packages, it can be important to update
the Ubuntu package list. This can be done by the following command:
\ubuntu{\$ sudo apt-get update}
\noindent For other operating systems, we refer to the specific
software documentation for installation instructions.

\section{Book outline}

This book covers the following material. We give an
introduction to brain physiology and brain imaging in
Chapter~\ref{chp:chp2}, as well as survey the software ecosystem
that will take us from MR images to numerical
simulation. In Chapter~\ref{chp:chp3}, our aim is to get up and
running quickly: we step through the entire pipeline from generating a
volume mesh from MR image data to solving our model
problem~\eqref{eq:diffusion} on this mesh. In Chapter~\ref{chp:chp4},
we cover more aspects of meshing including distinguishing between gray and
white matter, merging left and right hemispheres, and adding
parcellation labels. In Chapter~\ref{chap:dti}, we focus on diffusion
tensor imaging (DTI) and demonstrate how we can convert DTI data
to a numerical tensor field. Finally, in Chapter~\ref{chp:chp6}, we
bring everything from Chapters~\ref{chp:chp3} to \ref{chap:dti} together
to present a realistic simulation of anisotropic diffusion in
heterogeneous brain regions. 
 
