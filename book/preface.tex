%%%%%%%%%%%%%%%%%%%%%%preface.tex%%%%%%%%%%%%%%%%%%%%%%%%%%%%%%%%%%%%%%%%%
% sample preface
%
% Use this file as a template for your own input.
%
%%%%%%%%%%%%%%%%%%%%%%%% Springer %%%%%%%%%%%%%%%%%%%%%%%%%%

\preface

%% Please write your preface here
%% For most applied mathematicians, it comes rather as a shock that the
%% brain is not a unit square. There are many books on how to solve PDEs
%% on unit squares. We wanted to write this book to help the transition
%% for researchers used to defining meshes analytically, running
%% simulations in seconds, and never having to worry about storing mesh
%% labels -- to real geometries, noisy image data and messy and
%% inconsistent data formats. 


Questions surrounding the nature and fundamental biology of man are probably as old as 
mankind's ability for self reflection. Already in the ancient Egyptian writings
Edwin Smith Surgical Papyrus\footnote{The text is named after the dealer who bought it.}, 
1700 years BC, the pulsations
of the brain and the structure of its folds are described. Hippocrates, the father of medicine, states that
the brain is the seat of intelligence, while Aristotle was fascinated by and wrote about sleep and dream. 
Of course, the methods of investigation in the earlier times were crude and invasive.
Arguably, one of the profound medical achievements of our modern age is the advent of 
non-invasive imaging technologies.  The imaging revolution was birthed by 
Wilhelm Rontgen's discovery of X-rays already in 1895. A few years later  Marie Curie
managed to isolate radium and employ X-rays medically. 
Still, with these early breakthroughs, it was not until 
1950 the first positron emission tomography (PET) scan was developed and
later in 1967 that the first clinical
X-ray computed tomography (CT) scanner was set to use. After these successess there
has been an explosive developoment of different techniques and we now have
many at our disposal; in addition to PET and CT, 
the magnetic resonance imaging (MRI) have proven to be indispensable 
means for understanding the brain of living patients.  
The medical imaging field is now vast. For instance, 
the anual meeting of Radiological Society of North America (RSNA)  
hosts around 25 000 attendees, while  the corresponding number for the meeting in 
Society for Neuroscience (SfN) is 44 000. 

Meanwhile, maybe in particular in engineering applications, there
has been an explosive development of numerical methods and applications
using partial differential equations to model and understand in particular
physical phenomena.  The finite 
element method (FEM), in particular, was introduced in the 1960s for solving 
PDE on domains with complex geometries.  Significant work has been invested in 
the the construction of scalable, performant, and approachable software libraries 
for solving PDE using the FEM approach.  Today, we have many such libraries at 
our disposal; including Abaqus, COMSOL, deal.ii, DUNE, and {\fenics}, among numerous 
others. However, the generation of suitable, physiological finite element 
geometries for solving PDE on brain domains remains as a practical barrier in 
this regard and the impact of computational modeling on medical imaging 
and neuroscience has by far reached its full potential. 
Our aim with this book is to (at least partially) bridge the between
common tools in medical imaging and neuroscience with modern tools for numerical
solution of PDEs. 

The generation of physiological finite element meshes of the brain is not, in 
general, an easy task.  The sulcal, and gyral folds of the cortex are intricate, 
and the extracellular diffusion tensor, dictated largely by axonal white-matter 
bundles, is anisotropic and tortuous. Nevertheless, such features are 
essential for even the simplest, patient-specific PDE models of brain 
structural deformation and fluid dynamics.  Accurately capturing anatomical 
features is essential for a wide variety of problems of practical importance; 
in particular, for the understanding of the mechanisms underlying neurodegenerative 
pathology evolution.  This book stands at the gateway of these pressing problems.  


Herein, we guide the reader through a straightforward process for ascertaining 
the basic assets, i.e.~a finite element mesh and the extracellular diffusion 
tensor, from a set of patient MR images.  To do so, we introduce a novel software library, the 
Surface-Volume Meshing Toolkit (SVM-Tk), wrapping functionality from the broad 
array of capabilities provided by the Computational Geometry Algorithms Library 
(CGAL); thus providing an approachable set of features specifically for 
the brain modeling community.  Along the way, we will also employ the 
automatic-segementation capacity of the Freesurfer toolset; a gold-standard 
for MR image processing.  The marriage of mathematical modeling, clinical imaging, 
and numerical analysis is demonstrated by solving a simplified PDE model of 
anisotropic gadobutrol diffusion in the brain. 


We are deeply grateful to
the numerous colleagues who have provided advice, and guidance, along the
way as we commence our journey with you, the reader, into the exciting world
of mathematical brain modeling.


\kam{old text below}


Questions surrounding the nature and fundamental biology of man are as old as 
mankind's ability for self reflection.  Modalities of anatomical  
enquiry, and its methodology, hark back to the natural philosophers of 
the Greek empire.  The primary mode of early investigation, in these times, 
was careful dissection, and recording of observations.  To say that these early 
approaches were invasive is an understatement; observations, and postulations, 
regarding an in-vivo environment remained elusive.  Nevertheless, much was 
discovered and dissection remained the primary \textit{modus operandi} for 
centuries.  The early modern era, from the renaissance to the enlightenment, 
is demarcated by a rapid growth in human understanding; marked, especially, by 
significant advances in the mathematics, and mechanics, used to describe the natural 
world.  A primary achievement of the information age, in which we currently 
reside, is rapid rise of information processing technology.  In particular, 
the marriage of mathematics and computing.  This union has produced astounding 
advances in our understanding of the world. 

One of the profound medical achievements of the modern age is the advent of 
various clinical imaging technologies.  The imaging revolution was birthed by 
Wilhelm Rontgen's discovery of X-rays; allowing portions of the human in-vivo 
environment to be observed non-invasively.  We now have many imaging techniques 
at our disposal; in particular, positron emission tomography (PET), computed 
tomography (CT) and  magnetic resonance imaging (MRI) have proven to be indispensable 
means for understanding the brain of living patients.  The first PET scan of 
the brain emerged in the 1950s while CT and MR images of the brain were both 
produced in the 1970s; the medical imaging revolution was, officially, in full 
swing.  Since that time, we have seen an explosion of interest, research, and 
capability in clinical imaging technology applied to the brain; magnetic resonance 
angiography (MRA), functional MRI (fMRI), and diffusion tensor MRI (dMRI) have 
all been capitilized upon to enhance our understanding of brain function and 
pathology.  

Meanwhile, computing technology was rapidly accelerating in its pace of progress. 
Advancements in manufacturing processes were expeditiously increasing transistor 
density; producing ever-more-elaborate computing architectures capable of executing 
an astounding number of floating point operations per second.  Research into 
numerical methods, that can exploit computational power to solve partial 
differential equations (PDE) of previously-intractable complexity, spurred a 
resurgence of interest in mathematically modeling phenomena in the natural world.  The finite 
element method (FEM), in particular, was introduced in the 1960s for solving 
PDE on domains with complex geometries.  Significant work has been invested in 
the the construction of scalable, performant, and approachable software libraries 
for solving PDE using the FEM approach.  Today, we have many such libraries at 
our disposal; including Abaqus, COMSOL, deal.ii, DUNE, and {\fenics}, among numerous 
others. However, the generation of suitable, physiological finite element 
geometries for solving PDE on brain domains remains as a practical barrier in 
this regard.

The generation of physiological finite element meshes of the brain is not, in 
general, an easy task.  The sulcal, and gyral folds of the cortex are intricate, 
and the extracellular diffusion tensor, dictated largely by axonal white-matter 
bundles, is anisotropic and tortuous. Nevertheless, such features are 
essential for even the simplest, patient-specific PDE models of brain 
structural deformation and fluid dynamics.  Accurately capturing anatomical 
features is essential for a wide variety of problems of practical importance; 
in particular, for the understanding of the mechanisms underlying neurodegenerative 
pathology evolution.  This book stands at the gateway of these pressing problems.  


Herein, we guide the reader through a straightforward process for ascertaining 
the basic assets, i.e.~a finite element mesh and the extracellular diffusion 
tensor, from a set of patient MR images.  To do so, we introduce a novel software library, the 
Surface-Volume Meshing Toolkit (SVM-Tk), wrapping functionality from the broad 
array of capabilities provided by the Computational Geometry Algorithms Library 
(CGAL); thus providing an approachable set of features specifically for 
the brain modeling community.  Along the way, we will also employ the 
automatic-segementation capacity of the Freesurfer toolset; a gold-standard 
for MR image processing.  The marriage of mathematical modeling, clinical imaging, 
and numerical analysis is demonstrated by solving a simplified PDE model of 
anisotropic gadobutrol diffusion in the brain. 

We now, truly, stand on the shoulders of giants.  We are deeply grateful to the 
numerous colleagues who have provided advice, and guidance, along the way as we 
commence our journey with you, the reader, into the exciting world of mathematical 
brain modeling. 

%% Use the template \emph{preface.tex} together with the document class SVMono 
% (monograph-type books) or SVMult (edited books) to style your preface.

%% A preface\index{preface} is a book's preliminary statement, usually written 
% by the \textit{author or editor} of a work, which states its origin, scope, 
%purpose, plan, and intended audience, and which sometimes includes afterthoughts 
% and acknowledgments of assistance. 

%% When written by a person other than the author, it is called a foreword. 
% The preface or foreword is distinct from the introduction, which deals with 
%the subject of the work.

%% Customarily \textit{acknowledgments} are included as last part of the preface.
 
%We are very grateful to all those who have provided input on the book.

\vspace{\baselineskip}
\begin{flushright}\noindent
Oslo, Norway \hfill {\it Kent-Andr\'e Mardal}\\ 
Oxford, United Kingdom   \hfill {\it Marie E. Rognes}\\ 
             \hfill {\it Travis B. Thompson}\\ 
Jan, 2021    \hfill {\it Lars Magnus Valnes}\\ 
\end{flushright}


